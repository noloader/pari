\documentclass[11pt]{article}
%\overfullrule=5pt
\usepackage{amsmath}
\usepackage{amsfonts}
\usepackage{amssymb}

\DeclareMathOperator{\Tr}{Tr}
\DeclareMathOperator{\PSL}{PSL}
\DeclareMathOperator{\SL}{SL}
\DeclareMathOperator{\GL}{GL}
\newcommand{\Q}{{\mathbb Q}}
\newcommand{\Z}{{\mathbb Z}}
\newcommand{\R}{{\mathbb R}}
\newcommand{\C}{{\mathbb C}}
\newcommand{\z}{\zeta}
\renewcommand{\th}{\theta}
\newcommand{\G}{\Gamma}
\newcommand{\new}{\text{new}}

\def\kbd#1{{\tt #1}}

\begin{document}
\pagestyle{plain}

\title{Tutorial for Experimental Modular Forms in Pari/GP}
\author{Henri Cohen}

\maketitle

\smallskip

\section{Introduction}

The aim of this tutorial is to document the use of the additional functions
contained in the {\tt origin/henri-mf} branch of the modular form package,
which will be merged with the master branch at some time in the future.
Since these functions are mostly experimental and the branch is not the
master branch, it is highly plausible that the interface, including the
names of the functions, will change, as well as the results, but at least
it should give some idea of what is available.

\medskip

The main reasons for adding functionality to the main package are as follows:

\begin{enumerate}
\item Unless some quite strict conditions are satisfied, it is difficult
  to find the Fourier expansion of a modular form $f$ at an arbitrary cusp,
  and more generally to compute $f|_k\gamma$ for some $\gamma\in\SL_2(\Z)$
  (or even in $\GL_2^+(\Q)$).
\item As a consequence, it is not always easy to compute the action of
  the Atkin--Lehner operators and the Fricke involution.
\item It is also usually not easy to compute the numerical value of a
  modular form at a point close to the real axis, even after applying
  transformations to increase its imaginary part.
\item To compute spaces of modular forms of half-integral weight, a
  naive but workable idea is to divide by the standard weight $1/2$
  $\th$ function a form of integral weight, but for this one needs to know
  if a form is divisible by $\th$, which is equivalent to computing the
  valuation of a form at cusps.
\item This same idea can also be used to compute modular forms of weight
  $1$, but experimentation shows that this is orders of magnitude slower
  than the existing implementation using G.~Schaeffer's Hecke stability.
\end{enumerate}

\medskip

The (essentially unique) idea behind this additional package is a theorem
due to Borisov--Gunnells (not in this precise form, but it should be easy
to prove using their methods): in weight greater or equal to $3$,
any modular form is a linear combination of products of two (or one)
Eisenstein series; in weight $2$, only the modular forms whose
central value does not vanish are attained, and in weight $1$ there is not
enough room, but this does not really matter since one can always multiply
a form of weight $k\le2$ by some known Eisenstein series to obtain
a weight at least $3$.

Now it is easy (although rather tedious) to find $E|_k\gamma$ for an
Eisenstein series $E$, so in principle this allows to answer most of the
questions mentioned above, with one exception: evaluating a modular
form near the real axis can take time up to $N^{1/2}$ where $N$ is the
level, and unfortunately this is also true for Eisenstein series,
I do not see any way to do it faster except in special cases.

\section{The mfgaexpansion Function}

These functions will probably change, so some of them are not directly
accessible: if necessary read in the file ``\kbd{inst.gp}'' which is in the
distribution.

The function which does all the work is \kbd{mfgaexpansion} (name which
will certainly change): let $F$ be a modular form belonging to the modular form
space $mf$, and $\gamma\in \SL_2(\Z)$. The function

\centerline{\kbd{mfgaexpansion(mf,F,gamma,n)}}

computes the coefficients $a(0)$,...,$a(n)$ of the Fourier expansion at
infinity of $F|_k\gamma$. The result is of the form $[\alpha,w,V]$,
where $\alpha$ is a rational number, $w$ is the geometrical \emph{width}
of the cusp $\gamma(i\infty)$, and $V$ is a vector of $n+1$ coefficients
$[V[1],...,V[n+1]]$. The expansion is understood to be

  $$F|_k\gamma(\tau)=q^{\alpha}\sum_{0\le m\le n}V[m+1]q^{m/w}\;,$$

where as usual $q^u$ for nonintegral $u$ is understood to mean
$e^{2\pi iu\tau}$. The coefficients $V[m]$ are given as \emph{approximate}
complex numbers, except in some particularly simple cases where they are
given exactly. Note that $\alpha$ is only defined modulo $1/w$, so the
above notation is not unique, but $\alpha$ is guaranteed to be nonnegative.
In particular, the valuation of $F$ at the cusp $\gamma(i\infty)$ is
equal to $\alpha+m_0/w$, where $m_0$ is the smallest index $m\ge0$
(if it exists) such that $V[m+1]\ne0$.

This latter value can be obtained using the function
\kbd{mfcuspval(mf,F,cusp)}, where \kbd{cusp} must be specified as oo or
a rational number.

{\bf Important remark:} The result is (or should be!) independent of the
modular form space $mf$ to which $F$ belongs, although the exact way of
writing the expansion may be different. For efficiency reasons, it is of
course better to choose $mf$ with level as small as possible. On the other
hand it would be very unwise to \emph{remove} $mf$ from the notation,
first of course because it would have to recompute it each time, but more
importantly because the first time \kbd{mfgaexpansion} is called, some
black magic occurs which precomputes things which are put in $mf$ and need
not be recomputed later if the same $mf$ is used.

\medskip

We begin by a nontypical example in which the coefficients are given exactly:

\begin{verbatim}
? mf=mfinit([4,3,-4]); [B1,B2]=mfbasis(mf); [mfcoefs(B1,7),mfcoefs(B2,7)]
% = [[0, 1, 4, 8, 16, 26, 32, 48], [-1/4, 1, 1, -8, 1, 26, -8, -48]]
? mfgaexpansion(mf,B1,[1,0;2,1],10)
% = [1/2, 1, [1/4, 2, 13/2, 12, 73/4, 30, 85/2, 52, 145/2, 90, 96]~]
? mfgaexpansion(mf,B2,[1,0;2,1],10)
% = [1/2, 1, [1, -8, 26, -48, 73, -120, 170, -208, 290, -360, 384]~]
\end{verbatim}

In both cases the width is $1$, but $\alpha=1/2$ is nonzero so the
expansions are $q^{1/2}(1/4+2q+13/2q^2+\cdots)$ and
$q^{1/2}(1-8q+26q^2-\cdots)$ respectively. The fact that $\alpha\ne0$ is
typical of an \emph{irregular cusp}, which can occur only in odd weight, here
$3$.

Note also that (up to a trivial factor $4$, after all the basis $(B1,B2)$
is not canonical) the $n$th Fourier coefficient of $B2$ at the cusp $1/2$ (but
not at $i\infty$) is equal to $(-1)^n$ times that of $B1$.

Continuing the above example:

\begin{verbatim}
? mf=mfinit([16,3,-4]); mfgaexpansion(mf,B1,[1,0;2,1],10)
% = [1/2, 4, [1/4, 0, 0, 0, 2, 0, 0, 0, 13/2, 0, 0]~]
\end{verbatim}

In level $16$ the width of the cusp $1/2$ is equal to $4$, so the expansion
is in powers of $q^{1/4}$, of course identical to the one in level $4$.
  
\smallskip

Here is a more typical example:

\begin{verbatim}
? mf=mfinit([5,4],1); F=mfbasis(mf)[1]; mfcoefs(F,10)
% = [0, 1, -4, 2, 8, -5, -8, 6, 0, -23, 20]
? [al,w,V]=mfgaexpansion(mf,F,[0,-1;1,0],10)
% = [0, 5, [0, 0.040000...]]
\end{verbatim}

This tells us that $\alpha=0$, the width is $5$, and the expansion
begins $0.04000q^{1/5}+...q^{2/5}+...$.

The coefficients look very close to rational numbers, so we write:

\begin{verbatim}
? bestappr(V,1000)
% = [0, 1/25, -4/25, 2/25, 8/25, -1/5, -8/25, 6/25, 0, -23/25, 4/5]~
\end{verbatim}

and this is clearly equal to $1/25$ times the expansion at $i\infty$.
This is indeed trivial (at least up to sign) since $F$ must be an
eigenfunction of the Fricke involution $\tau\mapsto-1/(5\tau)$.

Note that if instead we do

\begin{verbatim}
? [al,w,V1] = mfgaexpansion(mf,F,[1,0;1,1],10)
% = [0, 5, [0, 0.01236...]]
\end{verbatim}

although this also corresponds to the same cusp $0$ (equivalent to $1$).
This illustrates the well-known fact that the Fourier expansion at a cusp
is not unique: we easily check that $V[m]=V1[m]/\z_5^{m-1}$, with
$\z_5=e^{2\pi i/5}$.
  
Remark: generally speaking, when the level is $N$, it is often
(but not always) the case that the coefficients belong to the cyclotomic
field $\Q(\z_N)$, and in that case they can be found by the \kbd{lindep}
command. For instance, to recognize the coefficient $V1[2]$ above:

\begin{verbatim}
? x=V1[2]; z=exp(2*Pi*I/5); vz=[1,z,z^2,z^3]; lindep(concat(-x,vz))
% = [25, 0, 1, 0, 0]~
\end{verbatim}

\medskip

An immediate application of the \kbd{mfgaexpansion} command is
\kbd{mfcuspval(mf,F,cusp)}:

\begin{verbatim}
? mf=mfinit([16,3,-4]); B=mfbasis(mf);
? vector(#B,i,mfcuspval(mf,B[i],1/2))
% = [1/2, 0, 0, 1/2, 1/2, 1/4, 1/4]
\end{verbatim}

In relation with $1/2$-integral weight which we will see below, we see that
$5$ of the $7$ forms in $B$ have valuation at $1/2$ at least equal to $1/4$,
which is that of $\th$, and this is easily seen to imply that they are
\emph{divisible} by $\th$. For the other two, we write

\begin{verbatim}
? mfgaexpansion(mf,B[2],[1,0;2,1],2)
% = [0, 4, [1/128*I, 0, 1/32]]
? mfgaexpansion(mf,B[3],[1,0;2,1],2)
% = [0, 4, [1/1024*I, 0.002...]]
? F = mflinear([B[2],B[3]],[1,-8]); mfcuspval(mf,F,1/2)
% = 1/4
\end{verbatim}

so that we have proved that the dimension of the space of forms divisible
by $\th$ is equal to $6$ (with basis $B[1]$, $B[4]$, $B[5]$, $B[6]$, $B[7]$,
and $B[2]-8B[3]$), hence the dimension of $M_{5/2}(\G_0(16))$ is also equal
to $6$ with an explicit basis. This can be checked by \kbd{mfdim} (see below)

\begin{verbatim}
? mfdim([16,5/2])
% = 6
\end{verbatim}

\smallskip

Finally, note that, as mentioned in the introduction, this function may fail
in weight $2$, and in that case simply multiply by your favorite form whose
expansion you know to get it in higher weight. Since the elliptic curve
over $\Q$ of smallest conductor with nonzero rank has conductor $37$, we
write:

\begin{verbatim}
? mf=mfinit([37,2],1); F=mfbasis(mf)[1];
? mfgaexpansion(mf,F,[0,-1;1,0],10)
  ***   at top-level: mfgaexpansion(mf,F,[0,-1;1,0],10)
  ***                 ^---------------------------------
  *** mfgaexpansion: sorry, mfeisendec when eisenspace not large enough
      is not yet implemented.
\end{verbatim}

A simple (although not very efficient) way to obtain the desired expansion
is as follows:

\begin{verbatim}
? E4=mfEk(4); F6=mfmul(F,E4); mf6=mfinit(F6); mf4=mfinit(E4);
? [al,w,V]=mfgaexpansion(mf6,F6,[0,-1;1,0],10);
? [al4,w4,V4]=mfgaexpansion(mf4,E4,[0,-1;1,0],10);
? V4w=vector(10,n,if((n-1)%(w/w4)==0,V4[(n-1)/(w/w4)],0));
? bestappr(Vec(Ser(V)/Ser(V4w)),1000)
% = [0, 0, -2/37, -4/37, 4/37, -2/37, 6/37, 0, 0, 8/37, 4/37]~
\end{verbatim}

In the above, since the width $w$ is equal to $37$ and we ask for only
$10$ coefficients, all the stuff concerning $E4$ can be suppressed, but
the above program is necessary if we ask for at least $37$ coefficients.

\medskip

Finally note the function \kbd{mfslash} (with the same syntax) that allows
to compute $f|_k\ga$ for $\ga\in\GL_2^+(\Q)$. For instance:

\section{Half-Integral Weight Modular Forms}

\subsection{Introduction}

Thanks to the \kbd{mfgaexpansion} function, it has been possible to write
a complete package for working with half-integral weight modular forms.
It is not very fast, but at least it gives this functionality to
\kbd{Pari/GP}.

The idea behind this implementation (which is transparent to the user)
has already been mentioned above: to find all modular forms of weight
$k+1/2$, find the vector space of forms of weight $k+1$ which are divisible
by $\th=1+2\sum_{n\ge1}q^{n^2}$. The necessary and sufficient condition for
divisibility of a form of weight $k+1$ by $\th$ is that its valuation
at all cusps of the form $A/C$ with $C\equiv2\pmod{4}$ be greater than or
equal to that of $\theta$, i.e., $1/4$; to have cusp forms, the form of
weight $k+1$ must be a cusp form and the valuations must be strictly
greater than $1/4$.

Recall that in the half-integral weight case the level $N$ must be
divisible by $4$, and the character $\chi$ must be an \emph{even}
character.

Inasmuch as possible we have made the commands transparent to the user, i.e.,
the basic commands (of course not those specific to half-integral weight)
are the same. Thus:

\begin{verbatim}
? apply(x->mfdim([24,7/2],x),[1,3,4])
% = [7, 6, 13]
? mf=mfinit([24,5/2],1); B=mfbasis(mf);
? for(i=1,#B,print(mfcoefs(B[i],10)))
[0, 2, -4, 14, -4, 0, -16, -20, 0, -6, 40]
[0, 0, 2, 2, 0, 0, -10, -8, 0, 0, 16]
[0, 0, 0, 12, 0, 0, -24, -24, 0, 0, 48]
\end{verbatim}

Note that the \kbd{mfdim} command is independent of the rest of the
package: it implements the Cohen--Oesterl\'e formulas together with the
use of the Serre--Stark theorem on modular forms of weight $1/2$.

\smallskip

Although there does exist a theory of old and newforms (at least in the
Kohnen $+$ space), we have not implemented new/old spaces (flags $0$ and $2$)
in the half-integral weight case. More annoying is the case of the space of
Eisenstein series (flag $3$): although we can give its dimension
(by an explicit formula), working out an explicit basis as in Weisinger's
thesis is doable but extremely painful (I have seen at least one reference),
so this is not implemented. In particular, the function \kbd{mfspace}
applied to a modular form of half-integral weight will almost always
return either $4$ (the full space, although the form may in fact belong
to the Eisenstein subspace, flag $3$), or $1$ (the cuspidal space, although
the form may in fact belong to the new space, flag $0$). In rare cases
(the functions \kbd{mfTheta} and \kbd{mfEH}) the function will return $3$.

\smallskip

Before continuing, note that because of the theta-multiplier, it will often
be the case that characters are multiplied by the character $\chi_{-4}$
(i.e., $(-4/n)$). For instance, continuing with the above example:

\begin{verbatim}
? mfparams(B[1])
% = [24, 5/2, 1]
? G=mfmul(B[1],B[1]); mfparams(G)
% = [24, 5, -4]
\end{verbatim}

\subsection{Specific Functions for Half-Integral Weight}

First, the Jacobi theta function of weight $1/2$:

\begin{verbatim}
? T=mfTheta(); mfcoefs(T,10)
% = [1, 2, 0, 0, 2, 0, 0, 0, 0, 2, 0]
? for(k=1,6,print1(mfparams(mfpow(T,k))))
[4, 1/2, 1] [4, 1, -4] [4, 3/2, 1] [4, 2, 1] [4, 5/2, 1] [4, 3, -4]
\end{verbatim}

Second, the Shimura lift:

\begin{verbatim}
? T5=mfpow(T,5); [mf,F,res]=mfShimura(T5,1); [mfparams(F),mfcoefs(F,7)]
% = [[1, 4, 1], [1/24, 10, 90, 280, 730, 1260, 2520, 3440]]
? [mf,F5,res]=mfShimura(T5,5); [mfparams(F),mfcoefs(F5,7)]
% = [[2, 4, 1], [-3/5, 112, 752, 3136, 5872, 14112, 21056, 38528]]
? T7=mfpow(T,7); [mf,F,res]=mfShimura(T7,-3); [mfparams(F),mfcoefs(F,7)]
% = [[1, 6, 1], [-5/9, 280, 9240, 68320, 295960, 875280, 2254560, 4706240]]
? mfparams(mf)
% = [2, 6, 1]
\end{verbatim}

Note a few things: first, as for the functions \kbd{mffromell} and
\kbd{mffromqf}, the result has 3 components: the space to which the result
belongs, the lift itself, and the coefficients of the lift on the given basis
of the space. The level of the lift is guaranteed to be a divisor of $N/2$
($N$ being the level of the 1/2-integral weight form), and this will always
be the level of the space mf. On the other hand, the level of the lift itself
may be lower, typically $N/4$, for instance when the form belongs to the
Kohnen plus splace. Note also that the level of the lift depends on the
discriminant taken for the lift. Finally, note that the sign of the
discriminant in weight $k+1/2$ must be $(-1)^k$.

\smallskip

Third, the Cohen--Hurwitz Eisenstein series of level $4$ and weight $r+1/2$:

\begin{verbatim}
? F1 = mfEH(1); mfcoefs(F1,12)
% = [-1/12, 0, 0, 1/3, 1/2, 0, 0, 1, 1, 0, 0, 1, 4/3]
? F2 = mfEH(2); mfcoefs(F2,10)
% = [1/120, -1/12, 0, 0, -7/12, -2/5, 0, 0, -1, -25/12, 0, 0, -2]
\end{verbatim}

\smallskip
  
Fourth, the Hecke operators $T(f^2)$ in half-integral weight (note
that the Hecke operators $T(n)$ for $n$ not a square are set to zero,
although they can also be defined when $n\mid N^\infty$).

\begin{verbatim}
% mfcoefs(mfhecke(F1,9), 10)
? = [-1/3, 0, 0, 4/3, 2, 0, 0, 4, 4, 0, 0]
\end{verbatim}

\smallskip

Some functions acquire additional functionalities thanks to the half-integral
weight functions. For instance, in the main package, \kbd{mffromqf} is
restricted to quadratic forms having even dimension. In this additional
package this restriction disappears:

\begin{verbatim}
? [mf,F,res]=mffromqf(2*matid(3)); mfcoefs(F,10)
% = [1, 6, 12, 8, 6, 24, 24, 0, 12, 30, 24]
? T=mfTheta(); T3=mfpow(T,3); mfisequal(F,T3)
% = 1
\end{verbatim}

Similarly for \kbd{mfetaquo}:

\begin{verbatim}
? F=mfetaquo(Mat([8,3])); mfcoefs(F,10)
% = [0, 1, 0, 0, 0, 0, 0, 0, 0, -3, 0]
? mfparams(F)
% = [64, 3/2, 8]
? mftobasis(mfinit([64,3/2,8]), F)
% = [0, 1, 2, -1/2, 0, -1/4, -1/2, -1/2, -1, 1/4]~
\end{verbatim}

\end{document}
